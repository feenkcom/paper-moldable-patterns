%% This file is cloned from `sample-sigconf.tex',
\documentclass[sigconf]{acmart}
%% \BibTeX command to typeset BibTeX logo in the docs
\AtBeginDocument{%
  \providecommand\BibTeX{{%
    Bib\TeX}}}
%% Rights management information.  This information is sent to you
%% when you complete the rights form.  These commands have SAMPLE
%% values in them; it is your responsibility as an author to replace
%% the commands and values with those provided to you when you
%% complete the rights form.
\setcopyright{acmlicensed}
\copyrightyear{2024}
\acmYear{2024}
\acmDOI{XXXXXXX.XXXXXXX}

% 29th European Conference on Pattern Languages of Programs
% https://www.europlop.net

%% These commands are for a PROCEEDINGS abstract or paper.
\acmConference[EuroPLoP '24]{29th European Conference on Pattern Languages of Programs}{July 3--7, 2024}{Kloster Irsee, Germany}
%%
%%  Uncomment \acmBooktitle if the title of the proceedings is different
%%  from ``Proceedings of ...''!
%%
%%\acmBooktitle{Woodstock '18: ACM Symposium on Neural Gaze Detection,
%%  June 03--05, 2018, Woodstock, NY}
\acmISBN{978-1-4503-XXXX-X/18/06}

\graphicspath{{figures/}}

\begin{document}

%% The "title" command has an optional parameter,
%% allowing the author to define a "short title" to be used in page headers.
\title{Moldable Development Patterns}

\author{Tudor G\^irba}
\affiliation{%
  \institution{feenk GmbH}
  \city{Wabern}
  \country{Switzerland}}
\email{tudor.girba@feenk.com}

\author{Oscar Nierstrasz}
\affiliation{%
  \institution{feenk GmbH}
  \city{Wabern}
  \country{Switzerland}}
\email{oscar.nierstrasz@feenk.com}

\renewcommand{\shortauthors}{G\^irba et al.}

\begin{abstract}
Moldable Development is a way to support decision-making by molding the development tools and environment to your problem, thus making the domain concepts visible, explorable, and explainable.
\end{abstract}

% \keywords{TODO}

%\received{20 February 2007}
%\received[revised]{12 March 2009}
%\received[accepted]{5 June 2009}

%%
%% This command processes the author and affiliation and title
%% information and builds the first part of the formatted document.
\maketitle

\section{Introduction}

\cite{Chis17a}


%% The next two lines define the bibliography style to be used, and
%% the bibliography file.
\bibliographystyle{ACM-Reference-Format}
\bibliography{moldablePatterns}


\end{document}
\endinput

